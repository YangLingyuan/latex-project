\title{Report for summer school CPS2018}
\author{
        \textsc{Lingyuan Yang}
            \qquad
        \\
        \normalsize
            \texttt{linyan@kth.se}
}
\date{\today}

\documentclass[12pt,twoside]{article}

\usepackage[paper=a4paper,dvips,top=1.5cm,left=1.5cm,right=1.5cm,
    foot=1cm,bottom=1.5cm]{geometry}


%\usepackage[T1]{fontenc}
%%\usepackage{pslatex}
\renewcommand{\rmdefault}{ptm} 
\usepackage{mathptmx}
\usepackage{amsmath}
\usepackage[scaled=.90]{helvet}
\usepackage{courier}

\usepackage[center]{titlesec}
\usepackage{bookmark}

\usepackage{fancyhdr}
\pagestyle{fancy}

%%----------------------------------------------------------------------------
%%   pcap2tex stuff
%%----------------------------------------------------------------------------
 \usepackage[dvipsnames*,svgnames]{xcolor} %% For extended colors
 \usepackage{tikz}
 \usetikzlibrary{arrows,decorations.pathmorphing,backgrounds,fit,positioning,calc,shapes}

%% \usepackage{pgfmath}	% --math engine
%%----------------------------------------------------------------------------
%% \usepackage[latin1]{inputenc}
\usepackage[utf8]{inputenc} % inputenc allows the user to input accented characters directly from the keyboard
\usepackage[english]{babel}
%% \usepackage{rotating}		 %% For text rotating
\usepackage{array}			 %% For table wrapping
\usepackage{graphicx}	                 %% Support for images
\usepackage{float}			 %% Suppor for more flexible floating box positioning
\usepackage{color}                       %% Support for colour 
\usepackage{mdwlist}
%% \usepackage{setspace}                 %% For fine-grained control over line spacing
%% \usepackage{listings}		 %% For source code listing
%% \usepackage{bytefield}                %% For packet drawings
\usepackage{tabularx}		         %% For simple table stretching
%%\usepackage{multirow}	                 %% Support for multirow colums in tables
\usepackage{dcolumn}	                 %% Support for decimal point alignment in tables
\usepackage{url}	                 %% Support for breaking URLs
\usepackage[perpage,para,symbol]{footmisc} %% use symbols to ``number'' footnotes and reset which symbol is used first on each page

%% \usepackage{pygmentize}           %% required to use minted -- see python-pygments - Pygments is a Syntax Highlighting Package written in Python
%% \usepackage{minted}		     %% For source code highlighting

 \usepackage{hyperref}		
\usepackage[all]{hypcap}	 %% Prevents an issue related to hyperref and caption linking
%% setup hyperref to use the darkblue color on links
 \hypersetup{colorlinks,breaklinks,
             linkcolor=darkblue,urlcolor=darkblue,
             anchorcolor=darkblue,citecolor=darkblue}

%% Some definitions of used colors
\definecolor{darkblue}{rgb}{0.0,0.0,0.3} %% define a color called darkblue
\definecolor{darkred}{rgb}{0.4,0.0,0.0}
\definecolor{red}{rgb}{0.7,0.0,0.0}
\definecolor{lightgrey}{rgb}{0.8,0.8,0.8} 
\definecolor{grey}{rgb}{0.6,0.6,0.6}
\definecolor{darkgrey}{rgb}{0.4,0.4,0.4}
%% Reduce hyphenation as much as possible
\hyphenpenalty=15000 
\tolerance=1000

%% useful redefinitions to use with tables
\newcommand{\rr}{\raggedright} %% raggedright command redefinition
\newcommand{\rl}{\raggedleft} %% raggedleft command redefinition
\newcommand{\tn}{\tabularnewline} %% tabularnewline command redefinition

%% definition of new command for bytefield package
\newcommand{\colorbitbox}[3]{%
	\rlap{\bitbox{#2}{\color{#1}\rule{\width}{\height}}}%
	\bitbox{#2}{#3}}

%% command to ease switching to red color text
\newcommand{\red}{\color{red}}
%%redefinition of paragraph command to insert a breakline after it
\makeatletter
\renewcommand\paragraph{\@startsection{paragraph}{4}{\z@}%
  {-3.25ex\@plus -1ex \@minus -.2ex}%
  {1.5ex \@plus .2ex}%
  {\normalfont\normalsize\bfseries}}
\makeatother

%%redefinition of subparagraph command to insert a breakline after it
\makeatletter
\renewcommand\subparagraph{\@startsection{subparagraph}{5}{\z@}%
  {-3.25ex\@plus -1ex \@minus -.2ex}%
  {1.5ex \@plus .2ex}%
  {\normalfont\normalsize\bfseries}}
\makeatother

\setcounter{tocdepth}{3}	%% 3 depth levels in TOC
\setcounter{secnumdepth}{5}
%%%%%%%%%%%%%%%%%%%%%%%%%%%%%%%%%%%%%%%%%%%%%%%%%%%%%%%%%%%%%%%%%%%%
%% End of preamble
%%%%%%%%%%%%%%%%%%%%%%%%%%%%%%%%%%%%%%%%%%%%%%%%%%%%%%%%%%%%%%%%%%%%

\renewcommand{\headrulewidth}{0pt}
\lhead{II2202, Fall 2017, Period 1-2}
%% or \lhead{II2202, Fall 2016, Period 1}
\chead{Final project report}
\rhead{\date{\today}}

\makeatletter
\let\ps@plain\ps@fancy 
\makeatother

\setlength{\headheight}{15pt}
\begin{document}

\maketitle

\centering
\section*{Executive summary}

This report is for the cyber security and privacy summer school 2018 in Trento. The name of our project
is EspioNo. We provide device to secure your private conversation from listening by smart speakers.
With a state-of-art technology disturbs the microphone but not human ear.
To introduce our project, this report is divided into 4 sections.
The first section will be focusing on the process of idea generation and optimization, as well as the technical basement.
The second section will be introducing our business canvas, detailize our business structure.
The third section will introduce our time line and expected cash flow. And the last section will evaluate the performance 
of myself both as a participant of summer school and a member of a team.



\clearpage

\selectlanguage{english}
\tableofcontents



\clearpage
\rr

\section{Problem and solution}
\subsection{Problem addressment}
From the point that Amazon post its 90 seconds 'Alexa Loses Her Voice' advertisment, people begin to fascinate with smart speakers 
for the convienience it introduce into our life. However, to interacte with the users, smart speaker has to keep listening to them. 
This feature arise doubt and panic on privacy issue. Some reports publised recently seems to confirm that such concern make sense. 
According to The New York Times \cite{nyctimesalexa} and Quartz \cite{quartzalexa}, alexa sometimes do record undesired conversation.
A study has been done across America showing that 54\% of the adults of the United States do not own a smart device in their home. 
6.4 milion people said specifically they would not buy a smart speaker due to privacy reasons. 6.4 million people. And what about 
the 20 million people who want to increase their privacy. In the united states alone, we have not even talked about the rest of the world. 
And this problem will only increase in the future. 



\section{Problem and solution}
\label{sec:Problem and solution}
\subsection{Problem addressment}
\subsection{Problem validation}
\subsection{Solution and validation}

\section{Business modelling and planning}
\label{sec:Business modelling and planning}
\subsection{Business modelling}
\subsection{Business planning}


\section{Business development process}
\label{sec:Business development process}

\section{Self evaluation}

\bibliography{II2202-report}
%%\bibliographystyle{IEEEtran}
\bibliographystyle{myIEEEtran}
%\appendix



\end{document}
